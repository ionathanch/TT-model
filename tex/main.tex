\documentclass[a4paper,UKenglish,cleveref,autoref,thm-restate]{lipics-v2021}

% \pdfoutput=1
% \hideLIPIcs

\usepackage[supertabular]{ottalt}
\let\newlist\relax
\let\renewlist\relax
\usepackage{enumitem,xspace,doi}
\usepackage{mathpartir,mathtools,stmaryrd}
\usepackage[flushmargin,multiple,para]{footmisc} % para spacing is weird and ugly

\newcommand{\citep}[1]{\cite{#1}}
\newcommand{\citet}[1]{\cite{#1}}
\newcommand{\publicrepo}{\url{https://github.com/ionathanch/TT-model}}
\newcommand{\lang}{\textsf{TT}\@\xspace}
\newcommand{\titlebreak}{\texorpdfstring{\\}{}}
\newcommand{\ie}{\textit{i.e.}\@\xspace}
\newcommand{\eg}{\textit{e.g.}\@\xspace}
\newcommand{\etal}{\textit{et al.}\@\xspace}
\newcommand{\vs}{\textit{vs.}\@\xspace}
\newcommand{\welltyped}{well-\hspace{0pt}typed\@\xspace}
\newcommand{\wellfounded}{well-\hspace{0pt}founded\@\xspace}
\newcommand{\wellfoundedness}{well-\hspace{0pt}foundedness\@\xspace}
\newcommand{\wellformedness}{well-\hspace{0pt}formedness\@\xspace}
\newcommand{\welldefinedness}{well-\hspace{0pt}definedness\@\xspace}
\newcommand{\crude}{crude-\hspace{0pt}but-\hspace{0pt}effective\@\xspace}

\newcommand{\footfile}[1]{%
  \raggedright
  \href{\publicrepo{}/tree/main/TT-model/#1}{\texttt{#1}}
}
\newcommand{\footfilethm}[2]{%
  \raggedright
  \href{\publicrepo{}/tree/main/TT-model/#1}{\texttt{#1}}\texttt{:#2}
}

\setlength{\fboxsep}{1.5pt}

\newlength{\punctwidth}
% \nspace{<punct>} is a negative space the size of <punct>
\newcommand{\nspace}[1]{%
  \settowidth{\punctwidth}{#1}%
  \hspace*{-\the\punctwidth}%
}
% \npunct{<punct>} treats <punct> as having no width,
% so that footnotes after it can stack on top
\newcommand{\npunct}[1]{#1%
  \nspace{#1}%
}
% \nsup{<sup>}{<punct>} shifts <punct> left by
% the width of <sup> as a superscript
% so that footnotes *before* it will stack on top
\newcommand{\nsup}[2]{%
  \nspace{$^{#1}$}%
  #2%
}

\setlength{\fboxsep}{2pt}
\setlength{\abovecaptionskip}{0.5\baselineskip}

\hypersetup{
  colorlinks=true,
  urlcolor=blue,
  citecolor=magenta
}
\urlstyle{tt}

\title{Bounded First-Class Universe Levels \titlebreak in Type Theory}
\titlerunning{Bounded First-Class Universe Levels}
\authorrunning{J. Chan, S. Weirich}
\Copyright{Jonathan Chan, Stephanie Weirich}
\ccsdesc{Theory of computation~Type theory}
\keywords{type theory, universes, universe polymorphism}
\hideLIPIcs

\author{Jonathan Chan}
  {University of Pennsylvania, Philadelphia, USA}
  {jcxz@seas.upenn.edu}
  {0000-0003-0830-3180}
  {}

\author{Stephanie Weirich}
  {University of Pennsylvania, Philadelphia, USA}
  {sweirich@seas.upenn.edu}
  {0000-0002-6756-9168}
  {}

\supplementdetails[subcategory={source code},
  swhid={swh:1:dir:} % see https://www.softwareheritage.org/
]{Software}{https://github.com/ionathanch/TT-model}

\inputott{rules}

\acknowledgements{hi \href{https://types.pl}{\texttt{types.pl}}!}

\begin{document}

\maketitle

\begin{abstract}
  abstract
\end{abstract}

\section{Introduction}

\subsection{Comparison to other work}

\section{A basic type theory with bounded first-class universe levels}

We consider with a Curry-style type theory \`a la Russell,
where terms have no type annotations,
and there is no separate typing judgement for well-formedness of types.
To keep the type theory minimal, it contains only dependent functions,
an empty type, predicative universes, and bounded universe levels.
By convention, we use $\ottnt{a}, \ottnt{b}, \ottnt{c}$ for terms,
$\ottnt{A}, \ottnt{B}, \ottnt{C}$ for types,
and $\ottnt{k}, \ell$ for level terms.
The syntax is presented in nominally in \cref{fig:syntax},
although the mechanization uses de Bruijn indexing.
We write single substitution of a variable $\ottmv{x}$ in a term $\ottnt{b}$ by another term $\ottnt{a}$
as $ \ottnt{b} [  \ottmv{x}  \mapsto  \ottnt{a}  ] $.
We also use $ \ottnt{A}  \to  \ottnt{B} $ as sugar for nondependent functions
$ \Pi  \ottmv{x}  \mathbin{:}  \ottnt{A}  \mathpunct{.}  \ottnt{B} $ where $\ottmv{x}$ does not occur in $\ottnt{B}$.

\begin{figure}
\begin{align*}
  i, j & \Coloneqq \texttt{<external universe levels>} \\
  x, y, z & \Coloneqq \texttt{<term variables>} \\
  a, b, c, A, B, C, k, \ell & \Coloneqq \ottmv{x} \mid \ottmv{i}
    \mid  \Pi  \ottmv{x}  \mathbin{:}  \ottnt{A}  \mathpunct{.}  \ottnt{B}  \mid  \lambda  \ottmv{x}  \mathpunct{.}  \ottnt{b}  \mid  \ottnt{b}  \gap  \ottnt{a} 
    \mid  \bot  \mid  \kw{absurd} \gap  \ottnt{b} 
    \mid  \kw{U} \gap  \ottnt{k}  \mid  \kw{Level}\texttt{<} \gap  \ell  \\
  \Gamma, \Delta & \Coloneqq  \cdot  \mid  \Gamma ,  \ottmv{x}  \mathbin{:}  \ottnt{A} 
\end{align*}
\caption{Syntax}
\label{fig:syntax}
\end{figure}

The type theory is parametrized over a cofinal woset of levels,
\ie a set of levels that are well founded, totally ordered,
and each have some strictly larger level;
these properties are required when modelling the type theory.
Instances of such sets include the naturals $0, 1, 2, \dots$,
as well as the naturals extended by one limit ordinal $\omega$
and its successors $\omega + 1, \omega + 2, \dots$.
We continue to use these concrete levels for our examples.
These metalevel levels are internalized directly in system as terms $\ottmv{i}$.

\begin{figure}
\begin{mathpar}
  \fbox{$ \mathop{\vdash}  \Gamma $} \qquad \fbox{$ \Gamma  \vdash  \ottnt{a}  \mathrel{:}  \ottnt{A} $} \qquad \fbox{$ \ottnt{a}  =  \ottnt{b} $} \hfill \\
  \inferrule[\ottdrulename{Nil}]{~}{ \mathop{\vdash}   \cdot  }
  \and
  \inferrule[\ottdrulename{Cons}]
    { \mathop{\vdash}  \Gamma  \and
      \Gamma  \vdash  \ottnt{A}  \mathrel{:}   \kw{U} \gap  \ottnt{k}  }
    %------------------%
    { \mathop{\vdash}   \Gamma ,  \ottmv{x}  \mathbin{:}  \ottnt{A}  }
  \and
  \inferrule[\ottdrulename{Var}]
    { \mathop{\vdash}  \Gamma  \and
      \ottmv{x}  \mathrel{:}  \ottnt{A}  \in  \Gamma }
    %-------------%
    { \Gamma  \vdash  \ottmv{x}  \mathrel{:}  \ottnt{A} }
  \and
  \inferrule[\ottdrulename{Pi}]
    { \Gamma  \vdash  \ottnt{A}  \mathrel{:}   \kw{U} \gap  \ottnt{k}   \and
       \Gamma ,  \ottmv{x}  \mathbin{:}  \ottnt{A}   \vdash  \ottnt{B}  \mathrel{:}   \kw{U} \gap  \ottnt{k}  }
    %--------------------------%
    { \Gamma  \vdash   \Pi  \ottmv{x}  \mathbin{:}  \ottnt{A}  \mathpunct{.}  \ottnt{B}   \mathrel{:}   \kw{U} \gap  \ottnt{k}  }
  \and
  \inferrule[\ottdrulename{Lam}]
    { \Gamma  \vdash   \Pi  \ottmv{x}  \mathbin{:}  \ottnt{A}  \mathpunct{.}  \ottnt{B}   \mathrel{:}   \kw{U} \gap  \ottnt{k}   \and
       \Gamma ,  \ottmv{x}  \mathbin{:}  \ottnt{A}   \vdash  \ottnt{b}  \mathrel{:}  \ottnt{B} }
    %-------------------------%
    { \Gamma  \vdash   \lambda  \ottmv{x}  \mathpunct{.}  \ottnt{b}   \mathrel{:}   \Pi  \ottmv{x}  \mathbin{:}  \ottnt{A}  \mathpunct{.}  \ottnt{B}  }
  \and
  \inferrule[\ottdrulename{App}]
    { \Gamma  \vdash  \ottnt{b}  \mathrel{:}   \Pi  \ottmv{x}  \mathbin{:}  \ottnt{A}  \mathpunct{.}  \ottnt{B}   \and
      \Gamma  \vdash  \ottnt{a}  \mathrel{:}  \ottnt{A} }
    %----------------------%
    { \Gamma  \vdash   \ottnt{b}  \gap  \ottnt{a}   \mathrel{:}   \ottnt{B} [  \ottmv{x}  \mapsto  \ottnt{a}  ]  }
  \and
  \inferrule[\ottdrulename{Mty}]
    { \Gamma  \vdash   \kw{U} \gap  \ottnt{k}   \mathrel{:}   \kw{U} \gap  \ell  }
    %-----------------------%
    { \Gamma  \vdash   \bot   \mathrel{:}   \kw{U} \gap  \ottnt{k}  }
  \and
  \inferrule[\ottdrulename{Abs}]
    { \Gamma  \vdash  \ottnt{A}  \mathrel{:}   \kw{U} \gap  \ottnt{k}   \and
      \Gamma  \vdash  \ottnt{b}  \mathrel{:}   \bot  }
    %-----------------%
    { \Gamma  \vdash   \kw{absurd} \gap  \ottnt{b}   \mathrel{:}  \ottnt{A} }
  \and
  \inferrule[\ottdrulename{Conv}]
    { \Gamma  \vdash  \ottnt{a}  \mathrel{:}  \ottnt{A}  \and
      \Gamma  \vdash  \ottnt{B}  \mathrel{:}   \kw{U} \gap  \ottnt{k}   \and
      \ottnt{A}  =  \ottnt{B} }
    %------------------%
    { \Gamma  \vdash  \ottnt{a}  \mathrel{:}  \ottnt{B} }
\end{mathpar}
%
\begin{mathpar}
  \inferrule[\ottdrulename{E-Beta}]{~}{   (  \lambda  \ottmv{x}  \mathpunct{.}  \ottnt{b}  )   \gap  \ottnt{a}   =   \ottnt{b} [  \ottmv{x}  \mapsto  \ottnt{a}  ]  } \and
  \inferrule[\ottdrulename{E-Refl}]{~}{ \ottnt{a}  =  \ottnt{a} } \and
  \inferrule[\ottdrulename{E-Sym}]{ \ottnt{a}  =  \ottnt{b} }{ \ottnt{b}  =  \ottnt{a} } \and
  \inferrule[\ottdrulename{E-Trans}]{ \ottnt{a}  =  \ottnt{b}  \and  \ottnt{b}  =  \ottnt{c} }{ \ottnt{a}  =  \ottnt{c} } \and
  \cdots
\end{mathpar}

\caption{Typing and selected equality rules (no universes or levels)}
\label{fig:typing:basic}
\end{figure}

We begin first with the basic rules that don't concern universes or levels in \cref{fig:typing:basic},
consisting of a context well-formedness judgement $ \mathop{\vdash}  \Gamma $,
a typing judgement $ \Gamma  \vdash  \ottnt{a}  \mathrel{:}  \ottnt{A} $, and an untyped definitional equality $ \ottnt{a}  =  \ottnt{b} $.
\Rref{Lam} explicitly includes well-typedness of a function's type as a premise
rather than merely well-typedness of the domain type $\ottnt{A}$
to guarantee that it lives at the same universe level as the codomain type $\ottnt{B}$,
which we need to prove our fundamental theorem.
The remaining rules are otherwise as expected.
We use $\beta$-conversion as our definitional equality,
and omit the usual congruence rules.

\begin{figure}
\begin{mathpar}
  \inferrule[\ottdrulename{Univ}]
    { \Gamma  \vdash  \ottnt{k}  \mathrel{:}   \kw{Level}\texttt{<} \gap  \ell  }
    %-----------------------%
    { \Gamma  \vdash   \kw{U} \gap  \ottnt{k}   \mathrel{:}   \kw{U} \gap  \ell  }
  \and
  \inferrule[\ottdrulename{Level<}]
    { \Gamma  \vdash   \kw{U} \gap  \ottnt{k_{{\mathrm{1}}}}   \mathrel{:}   \kw{U} \gap  \ell_{{\mathrm{1}}}   \and
      \Gamma  \vdash  \ottnt{k_{{\mathrm{0}}}}  \mathrel{:}   \kw{Level}\texttt{<} \gap  \ell_{{\mathrm{0}}}  }
    %---------------------------%
    { \Gamma  \vdash   \kw{Level}\texttt{<} \gap  \ottnt{k_{{\mathrm{0}}}}   \mathrel{:}   \kw{U} \gap  \ottnt{k_{{\mathrm{1}}}}  }
  \and
  \inferrule[\ottdrulename{Lvl}]
    { \mathop{\vdash}  \Gamma  \and
      \ottmv{i}  <  \ottmv{j} }
    %--------------------%
    { \Gamma  \vdash   \ottmv{i}   \mathrel{:}   \kw{Level}\texttt{<} \gap   \ottmv{j}   }
  \and
  \inferrule[\ottdrulename{Trans}]
    { \Gamma  \vdash  \ottnt{k_{{\mathrm{1}}}}  \mathrel{:}   \kw{Level}\texttt{<} \gap  \ottnt{k_{{\mathrm{2}}}}   \and
      \Gamma  \vdash  \ottnt{k_{{\mathrm{2}}}}  \mathrel{:}   \kw{Level}\texttt{<} \gap  \ottnt{k_{{\mathrm{3}}}}  }
    %----------------------%
    { \Gamma  \vdash  \ottnt{k_{{\mathrm{1}}}}  \mathrel{:}   \kw{Level}\texttt{<} \gap  \ottnt{k_{{\mathrm{3}}}}  }
  \and
  \inferrule[\ottdrulename{Sub}]
    { \Gamma  \vdash  \ottnt{A}  \mathrel{:}   \kw{U} \gap  \ottnt{k}   \and
      \Gamma  \vdash  \ottnt{k}  \mathrel{:}   \kw{Level}\texttt{<} \gap  \ell  }
    %--------------------%
    { \Gamma  \vdash  \ottnt{A}  \mathrel{:}   \kw{U} \gap  \ell  }
\end{mathpar}
\caption{Typing rules (universes and levels)}
\label{fig:typing:univ}
\end{figure}

The rules relating to universes and levels are given in \cref{fig:typing:univ}.
By \rref{Lvl}, we can view the type constructor $ \kw{Level}\texttt{<} \gap   \relax  $
as the internalization of the order on levels.
As the level type is combined with level bounding,
it is impossible to declare a level variable strictly less than itself.
The level type itself can be typed at any universe
regardless of what the bounding level is.
For example, we can construct a derivation for $  \cdot   \vdash   \kw{Level}\texttt{<} \gap   2    \mathrel{:}   \kw{U} \gap   0   $
solely knowing that $  \cdot   \vdash   2   \mathrel{:}   \kw{Level}\texttt{<} \gap   3   $ and $  \cdot   \vdash   \kw{U} \gap   0    \mathrel{:}   \kw{U} \gap   1   $.

\Rref{Trans} internalizes transitivity of the order on levels,
which is now required since levels are terms in general and not only concrete levels.
For example, we can construct a derivation for $   \ottmv{x}  \mathbin{:}   \kw{Level}\texttt{<} \gap   \omega    ,  \ottmv{y}  \mathbin{:}   \kw{Level}\texttt{<} \gap  \ottmv{x}    \vdash  \ottmv{x}  \mathrel{:}   \kw{Level}\texttt{<} \gap   \omega   $,
where the levels $\ottmv{x}, \ottmv{y}$ are variables.

\Rref{Sub} is a subsumption rule that permits lifting a type
from one universe to a higher universe.
This rule is weaker than a full cumulativity rule that accounts for
contravariance in the domain and covariance in the codomain of function types.
Therefore, for instance, $  \ottmv{f}  \mathbin{:}    \kw{U} \gap   2    \to   \kw{U} \gap   0      \vdash  \ottmv{f}  \mathrel{:}    \kw{U} \gap   1    \to   \kw{U} \gap   1    $ does \emph{not} hold.
Nonetheless, subsumption allows us to instead type the $\eta$-expansion
$  \ottmv{f}  \mathbin{:}    \kw{U} \gap   2    \to   \kw{U} \gap   0      \vdash    \lambda  \ottmv{x}  \mathpunct{.}  \ottmv{f}   \gap  \ottmv{x}   \mathrel{:}    \kw{U} \gap   1    \to   \kw{U} \gap   1    $.

Finally, \rref{Univ} asserts that a universe at level $\ottnt{k}$
lives in the universe at level $\ell$ when $\ottnt{k}$ is strictly bounded by $\ell$.
Allowing universes with general level terms and not just concrete levels
to be well typed is what permits typing level-polymorphic types.
For example, we can derive a typing judgement for
the following level-polymorphic identity function type:
$$  \cdot   \vdash    \Pi  \ottmv{x}  \mathbin{:}   \kw{Level}\texttt{<} \gap   \omega    \mathpunct{.}   \Pi  \ottmv{y}  \mathbin{:}   \kw{U} \gap  \ottmv{x}   \mathpunct{.}  \ottmv{y}    \to  \ottmv{y}   \mathrel{:}   \kw{U} \gap   \omega   $$
$ \kw{Level}\texttt{<} \gap   \omega  $ can be assigned an arbitrary type by \rref{Level},
$ \kw{U} \gap  \ottmv{x} $ has type $ \kw{U} \gap   \omega  $ by \rref{Univ} and \rref{Var},
and $\ottmv{y}$ can be assigned type $ \kw{U} \gap   \omega  $ transitively
via \rref{Trans,Var}.
Then the entire term has type $ \kw{U} \gap   \omega  $ by repeated application of \rref{Pi}.

\section{Metatheory}

The key properties of the type theory that we prove are
type safety and logical consistency.
Type safety is proven using standard syntactic methods
to show progress and preservation (\ie subject reduction).
Consistency follows from the fundamental soundness theorem
for a logical relation on closed types
where the empty type is interpreted as the empty set of terms.
The fundamental theorem also gives canonicity results for other types:
closed terms of the universe reduce to types,
and closed terms of level types reduce to concrete levels.

All results are mechanized in Lean.
While the mechanization uses de Bruijn indexing and simultaneous substitutions,
we omit the details of manipulating these substitutions,
continuing to state theorems in nominal form
to focus on the ideas rather than the minutiae.

\subsection{Type safety}

Our type safety theorem is stated in terms of a call-by-name evaluation strategy,
which we define shortly.
In essence, closed, well-typed terms evaluate (if they terminate) to values,
which are type formers and constructors,
defined as follows.
$$\ottnt{v} \Coloneqq \ottmv{i} \mid  \Pi  \ottmv{x}  \mathbin{:}  \ottnt{A}  \mathpunct{.}  \ottnt{B}  \mid  \lambda  \ottmv{x}  \mathpunct{.}  \ottnt{b}  \mid  \bot  \mid  \kw{U} \gap  \ottnt{k}  \mid  \kw{Level}\texttt{<} \gap  \ell $$

\subsubsection{Reduction and conversion}

\begin{figure}
\begin{mathpar}
  \inferrule[\ottdrulename{P-Beta}]
    { \ottnt{b}  \Rightarrow  \ottnt{b'}  \and
      \ottnt{a}  \Rightarrow  \ottnt{a'} }
    {   (  \lambda  \ottmv{x}  \mathpunct{.}  \ottnt{b}  )   \gap  \ottnt{a}   \Rightarrow   \ottnt{b'} [  \ottmv{x}  \mapsto  \ottnt{a'}  ]  }
  \and
  \inferrule[\ottdrulename{P-Var}]{~}{ \ottmv{x}  \Rightarrow  \ottmv{x} }
  \quad
  \inferrule[\ottdrulename{P-Mty}]{~}{  \bot   \Rightarrow   \bot  }
  \and
  \inferrule[\ottdrulename{P-Pi}]
    { \ottnt{A}  \Rightarrow  \ottnt{A'}  \and
      \ottnt{B}  \Rightarrow  \ottnt{B'} }
    {  \Pi  \ottmv{x}  \mathbin{:}  \ottnt{A}  \mathpunct{.}  \ottnt{B}   \Rightarrow   \Pi  \ottmv{x}  \mathbin{:}  \ottnt{A'}  \mathpunct{.}  \ottnt{B'}  }
  \and
  \inferrule[\ottdrulename{P-Lam}]
    { \ottnt{b}  \Rightarrow  \ottnt{b'} }
    {  \lambda  \ottmv{x}  \mathpunct{.}  \ottnt{b}   \Rightarrow   \lambda  \ottmv{x}  \mathpunct{.}  \ottnt{b'}  }
  \and
  \inferrule[\ottdrulename{P-App}]
    { \ottnt{b}  \Rightarrow  \ottnt{b'}  \and
      \ottnt{a}  \Rightarrow  \ottnt{a'} }
    {  \ottnt{b}  \gap  \ottnt{a}   \Rightarrow   \ottnt{b'}  \gap  \ottnt{a'}  }
  \and
  \inferrule[\ottdrulename{P-Abs}]
    { \ottnt{b}  \Rightarrow  \ottnt{b'} }
    {  \kw{absurd} \gap  \ottnt{b}   \Rightarrow   \kw{absurd} \gap  \ottnt{b'}  }
  \and
  \inferrule[\ottdrulename{P-Univ}]
    { \ottnt{k}  \Rightarrow  \ottnt{k'} }
    {  \kw{U} \gap  \ottnt{k}   \Rightarrow   \kw{U} \gap  \ottnt{k'}  }
  \and
  \inferrule[\ottdrulename{P-Level<}]
    { \ell  \Rightarrow  \ell' }
    {  \kw{Level}\texttt{<} \gap  \ell   \Rightarrow   \kw{Level}\texttt{<} \gap  \ell'  }
  \quad
  \inferrule[\ottdrulename{P-Lvl}]{~}{  \ottmv{i}   \Rightarrow   \ottmv{i}  }
\end{mathpar}
\caption{Parallel reduction rules}
\label{fig:par}
\end{figure}

Rather than working directly with $\beta$-reduction,
we use parallel reduction $ \ottnt{a}  \Rightarrow  \ottnt{b} $,
defined in \cref{fig:par},
and its reflexive, transitive closure $ \ottnt{a}  \Rightarrow^\ast  \ottnt{b} $,
into which call-by-name evaluation embeds.
We begin with simple lemmas about parallel reduction, stated without proof.

\begin{lemma}[Substitution (p.r.)] \label{lem:pars:subst}
  If $ \ottnt{a}  \Rightarrow^\ast  \ottnt{a'} $ and $ \ottnt{b}  \Rightarrow^\ast  \ottnt{b'} $,
  then $  \ottnt{b} [  \ottmv{x}  \mapsto  \ottnt{a}  ]   \Rightarrow^\ast   \ottnt{b'} [  \ottmv{x}  \mapsto  \ottnt{a'}  ]  $.
\end{lemma}

\begin{lemma}[Congruence (p.r.)] \label{lem:pars:cong}
  Parallel reduction is congruent,
  \eg if $ \ottnt{A}  \Rightarrow^\ast  \ottnt{A'} $ and $ \ottnt{B}  \Rightarrow^\ast  \ottnt{B'} $,
  then $  \Pi  \ottmv{x}  \mathbin{:}  \ottnt{A}  \mathpunct{.}  \ottnt{B}   \Rightarrow   \Pi  \ottmv{x}  \mathbin{:}  \ottnt{A'}  \mathpunct{.}  \ottnt{B'}  $,
  and so on for all terms.
\end{lemma}

\begin{lemma}[Construction (p.r.)] \label{lem:pars:cons}
  Analogous constructors of parallel reduction hold
  for its reflexive, transitive closure,
  \eg if $ \ottnt{b}  \Rightarrow^\ast  \ottnt{b'} $ and $ \ottnt{a}  \Rightarrow^\ast  \ottnt{a'} $,
  then $   (  \lambda  \ottmv{x}  \mathpunct{.}  \ottnt{b}  )   \gap  \ottnt{a}   \Rightarrow^\ast   \ottnt{b'} [  \ottmv{x}  \mapsto  \ottnt{a'}  ]  $.
\end{lemma}

\begin{lemma}[Inversion (p.r.)] \label{lem:pars:inv}
  If $ \ottnt{v}  \Rightarrow^\ast  \ottnt{c} $, then $\ottnt{c}$ is also a value of the same syntactic shape
  such that the reduction is congruent,
  \eg if $  \lambda  \ottmv{x}  \mathpunct{.}  \ottnt{b}   \Rightarrow^\ast  \ottnt{c} $, then $\ottnt{c}$ is syntactically equal to $ \lambda  \ottmv{x}  \mathpunct{.}  \ottnt{b'} $
  for some $\ottnt{b'}$ such that $ \ottnt{b}  \Rightarrow^\ast  \ottnt{b'} $.
\end{lemma}

Similarly, rather than working directly with definitional equality,
we use conversion $ \ottnt{a}  \Leftrightarrow  \ottnt{b} $,
which is defined in terms of parallel reduction.

\begin{definition}[Conversion]
  $ \ottnt{a}  \Leftrightarrow  \ottnt{b} $ iff there exists a $\ottnt{c}$ such that
  $ \ottnt{a}  \Rightarrow^\ast  \ottnt{c} $ and $ \ottnt{b}  \Rightarrow^\ast  \ottnt{c} $
\end{definition}

\begin{figure}
\begin{align*}
  \begin{aligned}
      (  \Pi  \ottmv{x}  \mathbin{:}  \ottnt{A}  \mathpunct{.}  \ottnt{B}  )  ^{ \mathsf{T} }  &\triangleq  \Pi  \ottmv{x}  \mathbin{:}   \ottnt{A} ^{ \mathsf{T} }   \mathpunct{.}   \ottnt{B} ^{ \mathsf{T} }   \\
      (  \lambda  \ottmv{x}  \mathpunct{.}  \ottnt{b}  )  ^{ \mathsf{T} }  &\triangleq  \lambda  \ottmv{x}  \mathpunct{.}   \ottnt{b} ^{ \mathsf{T} }   \\
      (  \kw{absurd} \gap  \ottnt{b}  )  ^{ \mathsf{T} }  &\triangleq   \kw{absurd} \gap  \ottnt{b}  ^{ \mathsf{T} } 
  \end{aligned}
  &
  \begin{aligned}
      (  \kw{U} \gap  \ottnt{k}  )  ^{ \mathsf{T} }  &\triangleq  \kw{U} \gap   \ottnt{k} ^{ \mathsf{T} }   \\
      (  \kw{Level}\texttt{<} \gap  \ell  )  ^{ \mathsf{T} }  &\triangleq  \kw{Level}\texttt{<} \gap   \ell ^{ \mathsf{T} }   \\
     \ottnt{a} ^{ \mathsf{T} }  &\triangleq a ~ \textit{otherwise}
  \end{aligned}
  &
  \begin{aligned}
      (  \ottnt{b}  \gap  \ottnt{a}  )  ^{ \mathsf{T} }  &\triangleq
    \begin{cases}
        \ottnt{b'} ^{ \mathsf{T} }  [  \ottmv{x}  \mapsto   \ottnt{a} ^{ \mathsf{T} }   ]  &\textit{if } \ottnt{b} \textit{ is }  \lambda  \ottmv{x}  \mathpunct{.}  \ottnt{b'}  \\
         \ottnt{b} ^{ \mathsf{T} }   \gap  \ottnt{a}  ^{ \mathsf{T} }  &\textit{otherwise}
    \end{cases} \\
    &
  \end{aligned}
\end{align*}
\caption{Complete development}
\label{fig:taka}
\end{figure}

Proving that conversion is transitive requires proving confluence for parallel reduction,
which we prove using the notion of complete development by Takahashi~\citep{takahashi}.
Complete development $ \ottnt{a} ^{ \mathsf{T} } $ is defined in \cref{fig:taka},
and is used to join parallel reductions and prove the diamond property.

\begin{lemma}[Completion (p.r.)] \label{lem:par:compl}
  If $ \ottnt{a}  \Rightarrow  \ottnt{b} $, then $ \ottnt{b}  \Rightarrow   \ottnt{a} ^{ \mathsf{T} }  $.
\end{lemma}

\begin{corollary}[Diamond (p.r.)]
  If $ \ottnt{a}  \Rightarrow  \ottnt{b} $ and $ \ottnt{a}  \Rightarrow  \ottnt{c} $,
  then there exists some $d$ such that $ \ottnt{b}  \Rightarrow  d $ and $ \ottnt{c}  \Rightarrow  d $.
  In particular, $d$ is $ \ottnt{a} ^{ \mathsf{T} } $,
  with the reductions given by \nameref{lem:par:compl}.
\end{corollary}

\begin{theorem}[Confluence (p.r.)] \label{lem:par:confl}
  If $ \ottnt{a}  \Rightarrow^\ast  \ottnt{b} $ and $ \ottnt{a}  \Rightarrow^\ast  \ottnt{c} $,
  then there exists some $d$ such that $ \ottnt{b}  \Rightarrow^\ast  d $ and $ \ottnt{c}  \Rightarrow^\ast  d $.
\end{theorem}

\begin{corollary}[Properties of conversion] \label{lem:conv}
  Conversion is reflexive, symmetric, transitive, substitutive, and congruent.
  Transitivity requires \nameref{lem:par:confl};
  the remaining properties are straightforward
  from the corresponding properties of parallel reduction.
\end{corollary}

Inversion on parallel reduction gives syntactic consistency and injectivity of conversion.

\begin{lemma}[Syntactic consistency]
  If $\ottnt{v_{{\mathrm{1}}}}$ and $\ottnt{v_{{\mathrm{2}}}}$ have different syntactic shapes,
  then $ \ottnt{v_{{\mathrm{1}}}}  \Leftrightarrow  \ottnt{v_{{\mathrm{2}}}} $ is impossible.
\end{lemma}

\begin{lemma}[Injectivity (conv.)] ~
  \begin{enumerate}[topsep=0pt]
    \item If $  \Pi  \ottmv{x}  \mathbin{:}  \ottnt{A_{{\mathrm{1}}}}  \mathpunct{.}  \ottnt{B_{{\mathrm{1}}}}   \Leftrightarrow   \Pi  \ottmv{x}  \mathbin{:}  \ottnt{A_{{\mathrm{2}}}}  \mathpunct{.}  \ottnt{B_{{\mathrm{2}}}}  $, then $ \ottnt{A_{{\mathrm{1}}}}  \Leftrightarrow  \ottnt{A_{{\mathrm{2}}}} $ and $ \ottnt{B_{{\mathrm{1}}}}  \Leftrightarrow  \ottnt{B_{{\mathrm{2}}}} $.
    \item If $  \kw{U} \gap  \ottnt{k_{{\mathrm{1}}}}   \Leftrightarrow   \kw{U} \gap  \ottnt{k_{{\mathrm{2}}}}  $, then $ \ottnt{k_{{\mathrm{1}}}}  \Leftrightarrow  \ottnt{k_{{\mathrm{2}}}} $.
  \end{enumerate}
\end{lemma}

Finally, definitional equality is equivalent to conversion,
which allows us to use them interchangeably later on.
The right-to-left direction is proven using induction on parallel reduction,
while the left-to-right direction is proven by induction on definitional equality,
using the various properties of conversion.

\begin{theorem} \label{lem:eq-conv}
  $ \ottnt{a}  =  \ottnt{b} $ iff $ \ottnt{a}  \Leftrightarrow  \ottnt{b} $.
\end{theorem}

\subsubsection{Subject reduction}

To prove subject reduction,
we begin first with some simple inversion properties on typing derivations,
proven directly by induction.

\begin{lemma}[Context well-formedness] \label{lem:wt:wf}
  If $ \Gamma  \vdash  \ottnt{a}  \mathrel{:}  \ottnt{A} $, then $ \mathop{\vdash}  \Gamma $.  
\end{lemma}

\begin{lemma}[Context well-typedness]
  If $ \mathop{\vdash}  \Gamma $ and $ \ottmv{x}  \mathrel{:}  \ottnt{A}  \in  \Gamma $,
  then there exists some $\ottnt{k}$ such that $ \Gamma  \vdash  \ottnt{A}  \mathrel{:}   \kw{U} \gap  \ottnt{k}  $.
\end{lemma}

\begin{lemma}[Regularity] \label{lem:wt:reg}
  If $ \Gamma  \vdash  \ottnt{a}  \mathrel{:}  \ottnt{A} $, then there exists some $\ottnt{k}$ such that
  $ \Gamma  \vdash  \ottnt{A}  \mathrel{:}   \kw{U} \gap  \ottnt{k}  $.
\end{lemma}

Next, we require weakening, substitution, and replacement lemmas.
They follow from stronger forms of these lemmas involving simultaneous renaming and substitution,
whose details we omit.

\begin{lemma}[Weakening (w.t.)] \label{lem:wt:weak}
  If $ \mathop{\vdash}  \Gamma $, $ \Gamma  \vdash  \ottnt{B}  \mathrel{:}   \kw{U} \gap  \ottnt{k}  $, and $ \Gamma  \vdash  \ottnt{a}  \mathrel{:}  \ottnt{A} $,
  then $  \Gamma ,  \ottmv{x}  \mathbin{:}  \ottnt{B}   \vdash  \ottnt{a}  \mathrel{:}  \ottnt{A} $, where $\ottmv{x}$ is not in $\ottnt{a}$ or $\ottnt{A}$.
\end{lemma}

\begin{proof}
  A renaming lemma is proven by induction on the typing derivation of $\ottnt{a}$,
  showing that applying well-scoped renamings preserves typing.
  Weakening is then the special case of a single renaming by $\ottmv{x}$ to avoid capture.
\end{proof}

\begin{lemma}[Substitution (w.t.)] \label{lem:wt:subst}
  If $ \Gamma  \vdash  \ottnt{b}  \mathrel{:}  \ottnt{B} $ and $  \Gamma ,  \ottmv{x}  \mathbin{:}  \ottnt{B}   \vdash  \ottnt{a}  \mathrel{:}  \ottnt{A} $,
  then $ \Gamma  \vdash   \ottnt{a} [  \ottmv{x}  \mapsto  \ottnt{b}  ]   \mathrel{:}   \ottnt{A} [  \ottmv{x}  \mapsto  \ottnt{b}  ]  $.
\end{lemma}

\begin{proof}
  A morphing lemma is proven by induction on the typing derivation of $\ottnt{a}$,
  showing that applying well-typed substitutions preserves typing.
  Substitution is then the special case of a single substitution of $\ottmv{x}$.
\end{proof}

\begin{lemma}[Replacement (w.t.)] \label{lem:wt:replace}
  If $ \ottnt{A}  =  \ottnt{B} $, $ \Gamma  \vdash  \ottnt{B}  \mathrel{:}   \kw{U} \gap  \ottnt{k}  $, and $  \Gamma ,  \ottmv{x}  \mathbin{:}  \ottnt{A}   \vdash  \ottnt{c}  \mathrel{:}  \ottnt{C} $,
  then $  \Gamma ,  \ottmv{x}  \mathbin{:}  \ottnt{B}   \vdash  \ottnt{c}  \mathrel{:}  \ottnt{C} $.
\end{lemma}

\begin{proof}
  The morphing lemma allows us to change the context from one to the other
  as long as there is a well-typed substitution between them.
  We can show that the identity substitution is such a substitution
  by proving $  \Gamma ,  \ottmv{x}  \mathbin{:}  \ottnt{B}   \vdash  \ottmv{x}  \mathrel{:}  \ottnt{A} $.
  This follows from \rref{Conv},
  \nameref{lem:wt:wf} for well-typedness of $\ottnt{A}$,
  and \nameref{lem:wt:weak} to weaken it.
\end{proof}

Finally, the proof of subject reduction requires many of the above lemmas.

\begin{theorem}[Subject reduction] \label{lem:preservation}
  If $ \ottnt{a}  \Rightarrow  \ottnt{b} $ and $ \Gamma  \vdash  \ottnt{a}  \mathrel{:}  \ottnt{A} $, then $ \Gamma  \vdash  \ottnt{b}  \mathrel{:}  \ottnt{A} $.
\end{theorem}

\begin{proof}
  By induction on the typing derivation of $\ottnt{a}$.
  The most complex case is when the reduction is \rref*{P-Beta},
  requiring \cref{lem:conv}, \nameref{lem:wt:subst},
  \nameref{lem:wt:replace}, and \nameref{lem:wt:reg}.
  Even so, the proof is standard,
  and the cases for the universe and level rules in \cref{fig:typing:univ}
  follow directly from the induction hypotheses.
\end{proof}

\subsubsection{Progress}

\begin{figure}
\begin{mathpar}
  \inferrule[\ottdrulename{N-Beta}]{~}
    {   (  \lambda  \ottmv{x}  \mathpunct{.}  \ottnt{b}  )   \gap  \ottnt{a}   \rightsquigarrow   \ottnt{b} [  \ottmv{x}  \mapsto  \ottnt{a}  ]  }
  \and
  \inferrule[\ottdrulename{N-App}]
    { \ottnt{b}  \rightsquigarrow  \ottnt{b'} }
    {  \ottnt{b}  \gap  \ottnt{a}   \rightsquigarrow   \ottnt{b'}  \gap  \ottnt{a}  }
  \and
  \inferrule[\ottdrulename{N-Abs}]
    { \ottnt{b}  \rightsquigarrow  \ottnt{b'} }
    {  \kw{absurd} \gap  \ottnt{b}   \rightsquigarrow   \kw{absurd} \gap  \ottnt{b'}  }
\end{mathpar}
\caption{Call by name reduction}
\label{fig:cbn}
\end{figure}

Our notion of evaluation in the progress and type safety theorems
is the reflexive, transitive closure  $ \ottnt{a}  \rightsquigarrow^\ast  \ottnt{b} $
of call-by-name (cbn) reduction $ \ottnt{a}  \rightsquigarrow  \ottnt{b} $,
defined in \cref{fig:cbn},
which reduces $\beta$-redexes and head positions.
A single step of cbn reduction embeds into
a single step of parallel reduction by induction.

\begin{lemma}
  If $ \ottnt{a}  \rightsquigarrow  \ottnt{b} $ then $ \ottnt{a}  \Rightarrow  \ottnt{b} $.
\end{lemma}

Because cbn reduces under eliminators for functions and proofs of falsehood,
we need to know what values of function and empty types are.
The following lemmas are proven by induction,
using inversion lemmas on typing derivations as needed.

\begin{lemma}[Canonical function values] \label{lem:canon:fun}
  If $ \Gamma  \vdash  \ottnt{v}  \mathrel{:}   \Pi  \ottmv{x}  \mathbin{:}  \ottnt{A}  \mathpunct{.}  \ottnt{B}  $,
  then $\ottnt{v}$ has the syntactic shape $ \lambda  \ottmv{x}  \mathpunct{.}  \ottnt{b} $ for some $\ottnt{b}$.
\end{lemma}

\begin{lemma}[Canonical false values] \label{lem:canon:false}
  There is no $\ottnt{v}$ such that $ \Gamma  \vdash  \ottnt{v}  \mathrel{:}   \bot  $.
\end{lemma}

Finally, we prove progress, which with subject reduction we show type safety:
closed, well-typed terms either reduce to a value, or must continue reducing.

\begin{theorem}[Progress] \label{lem:progress}
  If $  \cdot   \vdash  \ottnt{a}  \mathrel{:}  \ottnt{A} $, then either $\ottnt{a}$ is a value
  or $ \ottnt{a}  \rightsquigarrow  \ottnt{b} $ for some $\ottnt{b}$.
\end{theorem}

\begin{proof}
  By induction on the typing derivation of $\ottnt{a}$,
  using \cref{lem:canon:fun,lem:canon:false}
  in the cases for \rref{App,Abs}, respectively.
\end{proof}

\begin{theorem}[Type safety]
  If $  \cdot   \vdash  \ottnt{a}  \mathrel{:}  \ottnt{A} $ and $ \ottnt{a}  \rightsquigarrow^\ast  \ottnt{b} $,
  then either $\ottnt{b}$ is a value,
  or $ \ottnt{b}  \rightsquigarrow  \ottnt{c} $ for some $\ottnt{c}$.
\end{theorem}

\begin{proof}
  By induction on the reduction $ \ottnt{a}  \rightsquigarrow^\ast  \ottnt{c} $.
  The reflexive case holds by \nameref{lem:progress}.
  In the transitive case where $ \ottnt{a}  \rightsquigarrow  \ottnt{b} $ and $ \ottnt{b}  \rightsquigarrow^\ast  \ottnt{c} $,
  the goal holds by the induction hypothesis on the latter reduction,
  using \nameref{lem:preservation} for well typedness of $\ottnt{b}$
  from the former.
\end{proof}

\subsection{Consistency and canonicity}

To prove consistency and canonicity,
we use a logical relation to semantically interpret closed types as sets of closed terms;
these sets are backward closed under reduction,
so if a term reduces to something in the set, then it is also in the set.
The empty type is interpreted as the empty set,
universes as sets of terms that reduce to types,
and level types as sets of terms that reduce to concrete levels.
Consistency and canonicity then follow from the fundamental soundness theorem,
which states that if a term $\ottnt{a}$ has type $\ottnt{A}$,
then $\ottnt{a}$ is in the interpretation of $\ottnt{A}$.
The structure of the logical relation and the soundness proof
is adapted from the mechanization by Liu~\citep{lr-pearl},
and we will cover important details here,
especially as they pertain to universes and levels.

The logical relation is written as $ \mathopen{\llbracket}  \ottnt{A}  \mathclose{\rrbracket}_{ \ottmv{i} } \searrow  \ottnt{P} $,
where $\ottnt{A}$ is the type, $\ottnt{P}$ is the set of terms,
and $\ottmv{i}$ is the universe level of the type.
Because universes are interpreted as sets of types
which themselves have interpretations at a lower universe level,
to ensure that the interpretation is well defined,
the mechanization implements it as an inductive definition
parametrized by interpretations at lower levels,
then ties the knot by well-founded induction on levels.
For clarity and concision, we ignore these mechanization details
and present the logical relation in \cref{fig:lr:closed}
without worrying about well-foundedness.

\begin{figure}
\begin{mathpar}
  \inferrule[\ottdrulename{I-Mty}]{~}
    { \mathopen{\llbracket}   \bot   \mathclose{\rrbracket}_{ \ottmv{i} } \searrow   \varnothing  }
  \and
  \inferrule[\ottdrulename{I-Univ}]
    { \ottmv{j}  <  \ottmv{i} }
    %---------------------------------------%
  { \mathopen{\llbracket}   \kw{U} \gap   \ottmv{j}    \mathclose{\rrbracket}_{ \ottmv{i} } \searrow   \lbrace  \ottmv{z}  \mid   \exists  \ottnt{P}  \mathpunct{.}   \mathopen{\llbracket}  \ottmv{z}  \mathclose{\rrbracket}_{ \ottmv{j} } \searrow  \ottnt{P}    \rbrace  }
  \and
  \inferrule[\ottdrulename{I-Level<}]{~}
    { \mathopen{\llbracket}   \kw{Level}\texttt{<} \gap   \ottmv{j_{{\mathrm{1}}}}    \mathclose{\rrbracket}_{ \ottmv{i} } \searrow   \lbrace  \ottmv{z}  \mid   \exists  \ottmv{j_{{\mathrm{2}}}}  \mathpunct{.}    \ottmv{z}  \Rightarrow^\ast   \ottmv{j_{{\mathrm{2}}}}    \wedge   \ottmv{j_{{\mathrm{1}}}}  <  \ottmv{j_{{\mathrm{2}}}}     \rbrace  }
  \and
  \inferrule[\ottdrulename{I-Step}]
    { \ottnt{A}  \Rightarrow  \ottnt{B}  \and
      \mathopen{\llbracket}  \ottnt{B}  \mathclose{\rrbracket}_{ \ottmv{i} } \searrow  \ottnt{P} }
     %--------------%
    { \mathopen{\llbracket}  \ottnt{A}  \mathclose{\rrbracket}_{ \ottmv{i} } \searrow  \ottnt{P} }
  \and
  \inferrule[\ottdrulename{I-Pi}]
    { \mathopen{\llbracket}  \ottnt{A}  \mathclose{\rrbracket}_{ \ottmv{i} } \searrow  \ottnt{P_{{\mathrm{1}}}}  \and
       \forall  \ottmv{y}  \mathpunct{.}   \ottmv{y}  \in  \ottnt{P_{{\mathrm{1}}}}    \to   \exists  \ottnt{P_{{\mathrm{2}}}}  \mathpunct{.}   \ottnt{R} ( \ottmv{y} ,  \ottnt{P_{{\mathrm{2}}}} )    \\\\
       \forall  \ottmv{y}  \mathpunct{.}   \forall  \ottnt{P_{{\mathrm{2}}}}  \mathpunct{.}   \ottnt{R} ( \ottmv{y} ,  \ottnt{P_{{\mathrm{2}}}} )     \to   \mathopen{\llbracket}   \ottnt{B} [  \ottmv{x}  \mapsto  \ottmv{y}  ]   \mathclose{\rrbracket}_{ \ottmv{i} } \searrow  \ottnt{P_{{\mathrm{2}}}}  }
    %--------------------------------------------------------------------------------%
    { \mathopen{\llbracket}   \Pi  \ottmv{x}  \mathbin{:}  \ottnt{A}  \mathpunct{.}  \ottnt{B}   \mathclose{\rrbracket}_{ \ottmv{i} } \searrow   \lbrace  \ottmv{f}  \mid   \forall  \ottmv{y}  \mathpunct{.}    \forall  \ottnt{P_{{\mathrm{2}}}}  \mathpunct{.}    \ottnt{R} ( \ottmv{y} ,  \ottnt{P_{{\mathrm{2}}}} )   \to   \ottmv{y}  \in  \ottnt{P_{{\mathrm{1}}}}     \to    \ottmv{f}  \gap  \ottmv{y}   \in  \ottnt{P_{{\mathrm{2}}}}     \rbrace  }
\end{mathpar}
\caption{Logical relation for closed types}
\label{fig:lr:closed}
\end{figure}

\begin{mathpar}
  \inferrule[\ottdrulename{I-Pi'}]
    { \mathopen{\llbracket}  \ottnt{A}  \mathclose{\rrbracket}_{ \ottmv{i} } \searrow  \ottnt{P_{{\mathrm{1}}}}  \and
       \forall  \ottmv{y}  \mathpunct{.}   \ottmv{y}  \in  \ottnt{P_{{\mathrm{1}}}}    \to   \exists  \ottnt{P_{{\mathrm{2}}}}  \mathpunct{.}   \mathopen{\llbracket}   \ottnt{B} [  \ottmv{x}  \mapsto  \ottmv{y}  ]   \mathclose{\rrbracket}_{ \ottmv{i} } \searrow  \ottnt{P_{{\mathrm{2}}}}   }
    %--------------------------------------------------------------------------------%
    { \mathopen{\llbracket}   \Pi  \ottmv{x}  \mathbin{:}  \ottnt{A}  \mathpunct{.}  \ottnt{B}   \mathclose{\rrbracket}_{ \ottmv{i} } \searrow   \lbrace  \ottmv{f}  \mid   \forall  \ottmv{y}  \mathpunct{.}    \forall  \ottnt{P_{{\mathrm{2}}}}  \mathpunct{.}    (  \mathopen{\llbracket}   \ottnt{B} [  \ottmv{x}  \mapsto  \ottmv{y}  ]   \mathclose{\rrbracket}_{ \ottmv{i} } \searrow  \ottnt{P_{{\mathrm{2}}}}  )   \to   \ottmv{y}  \in  \ottnt{P_{{\mathrm{1}}}}     \to    \ottmv{f}  \gap  \ottmv{y}   \in  \ottnt{P_{{\mathrm{2}}}}     \rbrace  }
\end{mathpar}

\subsection{Attempts at proving normalization}

\section{Conclusion and future work}

\subsection{Extensions}

\citet{gen-univ,univ-poly}

\bibliographystyle{plainurl}
\bibliography{main.bib}

\end{document}